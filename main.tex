\documentclass[a4paper]{article} % this is used for comments
\usepackage[utf8]{inputenc}
%%% Дополнительная работа с математикой
\usepackage{amsmath,amsfonts,amssymb,amsthm,mathtools} % AMS
\usepackage{icomma} % "Умная" запятая: $0,2$ --- число, $0, 2$ --- перечисление
\usepackage[english,russian]{babel}

%% Номера формул
\mathtoolsset{showonlyrefs=true} % Показывать номера только у тех формул, на которые есть \eqref{} в тексте.

%% Шрифты
\usepackage{euscript}	 % Шрифт Евклид
\usepackage{mathrsfs} % Красивый матшрифт

\usepackage{physics}

%% Свои команды
\DeclareMathOperator{\sgn}{\mathop{sgn}}

%% Перенос знаков в формулах (по Львовскому)
\newcommand*{\hm}[1]{#1\nobreak\discretionary{}
{\hbox{$\mathsurround=0pt #1$}}{}}

\DeclareMathOperator{\Lin}{\mathrm{Lin}}
\DeclareMathOperator{\Linp}{\Lin^{\perp}}
\DeclareMathOperator*\plim{plim}
\DeclareMathOperator{\grad}{grad}
\DeclareMathOperator{\card}{card}
\DeclareMathOperator{\sgn}{sign}
\DeclareMathOperator{\sign}{sign}

\DeclareMathOperator*{\argmin}{arg\,min}
\DeclareMathOperator*{\argmax}{arg\,max}
\DeclareMathOperator*{\amn}{arg\,min}
\DeclareMathOperator*{\amx}{arg\,max}
\DeclareMathOperator{\cov}{Cov}
\DeclareMathOperator{\Var}{Var}
\DeclareMathOperator{\Cov}{Cov}
\DeclareMathOperator{\Corr}{Corr}
\DeclareMathOperator{\pCorr}{pCorr}
\DeclareMathOperator{\E}{\mathbb{E}}
\let\P\relax
\DeclareMathOperator{\P}{\mathbb{P}}

\newcommand{\cN}{\mathcal{N}}
\newcommand{\cU}{\mathcal{U}}
\newcommand{\cBinom}{\mathcal{Binom}}
\newcommand{\cBin}{\cBinom}
\newcommand{\cPois}{\mathcal{Pois}}
\newcommand{\cBeta}{\mathcal{Beta}}
\newcommand{\cGamma}{\mathcal{Gamma}}

\newcommand \R{\mathbb{R}}
\newcommand \N{\mathbb{N}}
\newcommand \Z{\mathbb{Z}}

\newcommand{\dx}[1]{\,\mathrm{d}#1} % для интеграла: маленький отступ и прямая d
\newcommand{\ind}[1]{\mathbbm{1}_{\{#1\}}} % Индикатор события
%\renewcommand{\to}{\rightarrow}
\newcommand{\eqdef}{\mathrel{\stackrel{\rm def}=}}
\newcommand{\iid}{\mathrel{\stackrel{\rm i.\,i.\,d.}\sim}}
\newcommand{\const}{\mathrm{const}}

% вместо горизонтальной делаем косую черточку в нестрогих неравенствах
\renewcommand{\le}{\leqslant}
\renewcommand{\ge}{\geqslant}
\renewcommand{\leq}{\leqslant}
\renewcommand{\geq}{\geqslant}


\title{Homework}
\author{Андрей Ильин, БЭК182}


\begin{document}


\maketitle



\textbf{Midterm exam 2017-2018}

AECBA BBCEB BBCCA BBCCC ABCBA EA?AC

\begin{enumerate}

    \item

    По свойству дисперсий:

    \[ \Var(X) = \E(X^2) - (\E(X))^2 \]

    \[ (\E(X^2) = \Var(X) + \E(X))^2 = 10\]
    
    $\P(X^2 \geq 100)$ похоже на неравенство Маркова. $X^2 \geq 0$ всегда $\Rightarrow$ используем эквивалент формуле из следствия неравенства.
    
    Получаем верхнюю границу диапозона:

    \[\P(X^2 \geq 100) \leq \frac{\E(X^2)}{100} = 0.1  \Rightarrow [0, 0.1]\]

    Ответ: A

    \item

    Так как $\xi$ имеет распределение Пуассона, то:

    \[ \E(\xi) =\lambda \]

    \[ \Var\xi) =\lambda \]
    
    По свойству дисперсий:
    
    \[ \Var(\xi) = \E(\xi^2) - (\E(\xi))^2 \]
    \[ \E(\xi^2) = \Var(\xi) + (\E(\xi))^2 = \lambda + \lambda^2 = \lambda \cdot (1 + \lambda)\]

    Ответ: E

    \item
    
    По формуле:

    \[ \Corr(X+Y,Y) = \frac{\Cov(X+Y,Y)}{\sqrt{\Var(X+Y)\cdot \Var(Y)}} = \frac{6}{\sqrt{7\cdot 9}} = \frac{2}{\sqrt{7}} \]

    \[ \Var(X+Y) = \Var(X) + \Var(Y) + 2 \cdot \Cov(X,Y) = 4 + 9 + 2 \cdot (-3) = 7 \]

    \[\Cov(X+Y,Y) = \Cov(X,Y) + \Cov(Y,Y) = -3 + 9 = 6\]

    Ответ: C

    \item
    
    Функция плотности для любой случайной величины с нормальным распределением:
    
    \[ f(x)=\frac{1}{\sqrt{2\pi\sigma^2}} e^{-\frac{(x-\mu)^2}{2\sigma^2}} \]
    
    У случайных величин со стандартным нормальным распределением $\sigma=1$ и $\mu=0$. При подстановке значений получаем ответ B

    Ответ: B

    \item

    Так как величина распределена равномерно по площади треугольника с координатами точек $(0;0), (0;4), (2;0)$:

    \[f_{X,Y}(1,1) = \frac{1}{S} = \frac{1}{\frac{1}{2}\cdot 4\cdot 2} = \frac{1}{4}\]

    Ответ: A

    \item
    
    По определению события A, B и C независимы в совокупности, если:
    
    \[ \P(ABC) = \P(A)\P(B)\P(C) \]

    Ответ: B

    \item

    При построении графика функции плотности $\xi$ получается прямоугольник с высотой $\frac{1}{4}$
    Площадь всего прямоугольника (от 0 до 4) должна быть равна одному, т. к. интеграл от $-\infty$ до $+\infty$ от функции плотности равен 1 по определению.
    
    \[\P({\xi\in[3,6]}) = \frac{1}{4}\]

    Ответ: B

    \item

    X, Y - случайные величины

    \[\P(X=-5) = \dots = \P(X=5) = \frac{1}{11}\]

    \[\P(Y=-1) = \P(Y=0) = \P(Y=1) = \frac{1}{3}\]

    Для $X + Y^2 = 2$ имеется всего три случая:

    \[Y = -1 \Rightarrow X = 1\]

    \[Y = 0 \Rightarrow X = 2\]

    \[Y = 1 \Rightarrow X = 1\]

    Случайные величины независимые $\Rightarrow \P(X + Y^2 = 2) = \frac{1}{11} \cdot\frac{1}{3} \cdot 3 = \frac{1}{11}$

    Ответ: C

    \item
    
    Зная, что один сектор равен $\frac{\pi}{3}$ найдем число секторов:

    \[\frac{2\pi}{\frac{\pi}{3}} = 6\]
    
    Все точки точки круга равновероятны, следовательно:

    \[\P(\text{<<попадет в красный>>}) = \frac{1}{6}\]

    Ответ: E

    \item
    
    По формуле:

    \[\P(A\cup B) = \P(A) + \P(B) - \P(A\cap B)\]

    \[0.6 = 0.3 + \P(B) - 0.2\]
    
    Соответственно:

    \[\P(B) = 0.5\]

    Ответ: B

    \item
    
    Используя свойства дисперсии:

    \[\Var(2X - Y + 1) = 4\cdot \Var(X) + \Var(Y) - 4 \cdot \Cov(X,Y) \]
    
    \[\Var(2X - Y + 1) = 4 \cdot 4 + 9 - 4 \cdot (-3) = 37\]

    Ответ: B

    \item
    
    Согласно ЗБЧ:

    \[\plim _{n\rightarrow +\infty}\frac{X_{1}^2 + \dots + X_{n}^2}{n} = \E(X^2) = \Var(X) +(\E(X))^2 = 1\]

    Ответ: B

    \item

    Условная функция плотности:

    \[f\Bigr(x\mid y=\frac{1}{2}\Bigl) = \frac{f(x,\frac{1}{2})}{f_{y}(\frac{1}{2})} = \frac{6x\cdot\frac{1}{4}}{\frac{3}{4}} = 2x\]

    \[f_{y}(y) = \int_0^1 6\cdot x\cdot y^2 dx = \left.3 \cdot x^2 \cdot y^2\right|_0^1  = 3\cdot y^2, y \in [0;1] \]
    
    \[f_{y}\Bigr(\frac{1}{2}\Bigl) = 3 \cdot \Bigr(\frac{1}{2}\Bigl)^2 = \frac{3}{4}\]

    Ответ: C

    \item

    В условии пропущено, чему равно n. Без этого можно подогнать любой ответ. Пусть $n = 100$.

    $X_{1}, X_{2}, \dots$ независимы и одинаково распределены

    \[\E(X_i) = 4\]

    \[\Var(X_i) = 100\]

    \[\P(\overline{X_n} \leq 5) - ?\]

    \[\overline{X} \sim \N\Bigr(4,\frac{100}{100}\Bigl)\]
    
    По таблице для нормального распределения:

    \[\P\Bigr(\frac{\overline{X} - 4}{\sqrt{1}} \leq \frac{5-4}{1}\Bigl) = \P(\Z\leq 1) = 0.8413\]
    
    Ответ: C

    \item
    
    Используя свойства ковариации:

    \[\Cov(X+2Y, 2X + 3) = \Cov(X+2Y, 2X) = \Cov(X, 2X) + \Cov(2Y, 2X) \]
    \[\Cov(X+2Y, 2X + 3) = 2 \cdot \Cov(X,X) + 4 \cdot \Cov(X,Y) = 2 \cdot 4 + 4 \cdot (-3) = -4\]

    Ответ: A

    \item
    
    Используя свойства математического ожидания:

    \[\E((X-1)Y) = \E(XY - Y) = \E(XY) - \E(Y) = \Cov(X,Y) + \E(X)\cdot\E(Y) -\E(Y)\] 
    \[\E((X-1)Y) = -3 + (-2) - 2 = -7\]

    Ответ: B

    \item

    $X_{i} = 1$, если <<6>>. $\P(X_{i} = 1) = \frac{1}{6}$

    $X_{i} = 0$, иначе. $\P(X_{i} = 0) = \frac{5}{6}$

    \[\P(X_{1} + X_{2} = 1) = \P(X_{1} = 0, X_{2} = 1) + \P(X_{1} = 1, X_{2} = 0) \]
    \[\P(X_{1} + X_{2} = 1) = \frac{5}{6} \cdot \frac{1}{6} + \frac{1}{6} \cdot \frac{5}{6} = \frac{10}{36}\]
    
    \[\P(X_{1} = 0\mid X_{1} + X_{2} = 1) = \frac{\P(X_{1} = 0\cap X_{1} + X_{2} = 1)}{X_{1} + X_{2} = 1}) = \frac{1}{2}\]
    
    Аналогично для $\P(X_{1} = 0\mid X_{1} + X_{2} = 1)$
    
    Условный закон $X_{1}$ совпадает с распределением Бернулли с $p = \frac{1}{2}$
    
    Ответ: B

    \item

    X и Y - независимые случайные величины

    \[X + Y \sim \N(\E(X) + \E(Y), \Var(X) + \Var(Y))\]

    \[X + Y \sim \N(3,7)\]
    
    Используя таблицу для нормального распределения:

    \[\P(X + Y\leq 3) = \P\Bigr(\frac{X+Y-3}{\sqrt{7}} < \frac{3-3}{\sqrt{7}}\Bigl) = (\Z\leq 0) = \frac{1}{2}\]

    Ответ: C

    \item

    5 кнопок:

    $i=1,2,3 \quad \P(x_i = 6) =\frac{1}{2}$ (честные кубики)

    $i=4 \quad \P(x_i = 6) =\frac{1}{2}$ (с увеличенной вероятностью выпадения 6)

    $i=5 \quad \P(x_i = 6) =\frac{1}{10}$ (с увеличенной вероятностью выпадения 1)

    \[\P(i=1,2,3\mid \text{<<6>>}) = \frac{\P(i=1,2,3\cap \text{<<6>>})}{\P(\text{<<6>>})} = \frac{\frac{3}{5}\cdot \frac{1}{6}}{\frac{1}{5}\cdot \frac{1}{6} + \frac{1}{5}\cdot \frac{1}{6} + \frac{1}{5}\cdot \frac{1}{6} + \frac{1}{5}\cdot \frac{1}{2} + \frac{1}{5}\cdot \frac{1}{10}} = \frac{\frac{3}{30}}{\frac{11}{50}} = \frac{5}{11} \]

    Ответ: C

    \item


    E) Должна быть симметричной

    D) Не может быть отрицательной

    A) $1\cdot1 - 2\cdot2 < 0$

    B) $1\cdot9 - 4\cdot4 < 0$

    C) $9\cdot6 - 7\cdot7 > 0$

    Ответ: C

    \item
    
    Используя свойства математического ожидания:

    \[\E(\alpha X + (1 - \alpha)Y) = \alpha \E(X) + (1-\alpha) \E(Y) = -\alpha + 2\cdot(1-\alpha) = 0\]
    \[2 - 3\cdot\alpha = 0\]
    \[\alpha = \frac{2}{3}\]

    Ответ: A

    \item
    
    $\xi$ имеет биноминальное распределение

    \[\P(\xi = 0) = (1-p)^n = \Bigr(\frac{1}{4}\Bigl)^2 = \frac{1}{16}\]

    Ответ: B

    \item
    
    Распределение Пуассона с $\lambda = 4$
    
    \[\P(x=k) = \lambda^k\cdot\frac{e^{-\lambda}}{k!}\]
    \[\P(X\geq1) = 1 - \P(k=0) = 1 - e^{-4}\]

    Ответ: C

    \item
    
    $\xi$ имеет распределение Бернулли

    \[\E(\xi^2) = \Var(\xi) + (\E(\xi))^2 = p\cdot (1-p) + p^2 = p\]

    Ответ: B

    \item
    
    $\xi$ имеет экспоненциальное распределение

    \[\E(\xi) = \frac{1}{\lambda}\]
    \[\Var(\xi) = \frac{1}{\lambda^2}\]

    \[\E(\xi^2) = \Var(\xi) + (\E(\xi))^2 = \frac{1}{\lambda^2} + \frac{1}{\lambda^2} = \frac{2}{\lambda^2}\]

    Ответ: A

    \item

    Зная, что один сектор равен $\frac{\pi}{3}$ найдем число секторов:

    \[\frac{2\pi}{\frac{\pi}{3}} = 6\]

    \[\P(\text{<<попадет в красный>>}) = \P(\text{<<попадет в синий>>}) =\frac{1}{6}\]

    Невозможно попасть одновременно в две доли $\Rightarrow$ событие A и событие B несовместны
    
    Ответ: E

    \item

    \[\E(XY) = \int_0^1\int_0^1 x\cdot y \cdot 6 \cdot x \cdot y^2 dxdy = \left.\int_0^1 2\cdot x^3 \cdot y^3 \right|_0^1 dy = \left.\frac{2\cdot y^4}{4}\right|_0^1 = \frac{1}{2}\]

    Ответ: A

    \item
    
    Используя свойства дисперсии:
    \[\Var(\alpha X + (1-\alpha) Y) = \alpha^2 \Var(X) + (1-\alpha)^2 \Var(Y) + 2\cdot \Cov(X,Y)\cdot\alpha\cdot(1-\alpha)\]
    \[\Var(\alpha X + (1-\alpha) Y) = 4\cdot\alpha^2 + 9\cdot(1-\alpha)^2 - 6\cdot\alpha\cdot(1-\alpha) = 4\cdot\alpha^2 + 9 - 18\cdot\alpha + 9\cdot\alpha^2 - 6\cdot\alpha + 6\cdot \alpha^2\]
    \[\Var(\alpha X + (1-\alpha) Y) = 19\cdot\alpha^2 - 24\cdot\alpha + 9\]
    Находим точку минимума:
    \[\alpha^{*} = \frac{24}{38} =\frac{12}{19}\]

    Ответ: F (нет верного ответа)

    \item

    \[\P(\text{<<без багажа>>}) = \frac{1}{4}\]
    \[\P(\text{<<с рюкзаком>>}\mid\text{<<без багажа>>}) = 0.5\]
    \[\P(\text{<<с рюкзаком>>}\mid\text{<<c багажом>>}) = \frac{55}{150}\]

    \begin{multline*}
        \P(\text{<<без рюкзака>>}) = \P(\text{<<без рюкзака>>}\mid\text{<<без багажа>>})\P(\text{<<без багажа>>}) +\\+ \P(\text{<<без рюкзака>>}\mid\text{<<с багажом>>})\P(\text{<<с багажом>>}) = \frac{1}{2}\cdot\frac{1}{4} + \frac{95}{150} \cdot\frac{3}{4} = 0.6
    \end{multline*}
   

    Ответ: A

    \item
    По условию:
    \[\E(X) = 2\]
    \[\Var(X) = 6\]

    $\P(|X - 2| \leq10)$ - похоже на неравенство Чебышева, но знак неравенства в другую сторону

    \[\P(|X-2| \geq 10) \leq \frac{\Var(X)}{100}\]

    \[\P(|X-2| \leq 10) \geq 1 - \frac{\Var(X)}{100} = 0.94\]

    Ответ: C

\end{enumerate}

\end{document}
